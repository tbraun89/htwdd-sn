\section{Positionierungs Algorithmen}
\label{sec:algorithmen}

\begin{frame}
  \frametitle{Algorithmen\footnote{Gholami, M. R. (2011). Positioning Algorithms for
      WirelessSensor Networks. Chalmers University of Technology.}}

  \begin{center}
    \includegraphics[scale=0.35]{img/algo_1}
  \end{center}
\end{frame}

\begin{frame}
  \frametitle{Triangulation}

  \begin{center}
    \includegraphics[scale=0.2]{img/triang}\\~\\

    $c = \sqrt{(B_{x)} - A_{x})^2 + (B_{y} - A_{y})^2}$\\
    $\varphi = arctan(\frac{B_{y} - A_{y}}{B_{x}} - A_{x})$\\
    $b = c \cdot \frac{sin(\beta)}{sin(\alpha + \beta)}$\\~\\

     \begin{tabular}{rr}
       $C_{x} = A_{x} + b \cdot cos(\varphi)$ & $C_{y} = A_{x} + b \cdot sin(\varphi)$ \\ 
     \end{tabular}
  \end{center}
\end{frame}

\begin{frame}
  \frametitle{Trilateration}

  \begin{center}
    \includegraphics[scale=0.2]{img/trilat}\\~\\

    $
    \begin{pmatrix}
      P_{x} \\
      P_{y}
    \end{pmatrix}
    =
    H^{-1} \cdot z
    $\\~\\
    $
    H = 
    \begin{bmatrix}
      2 \cdot x_{1} - 2 \cdot x_{2} & 2 \cdot y_{1} - 2 \cdot y_{2} \\
      2 \cdot x_{1} - 2 \cdot x_{3} & 2 \cdot y_{1} - 2 \cdot y_{3}
    \end{bmatrix}
    $\\~\\
    $
    z = 
    \begin{pmatrix}
      r_{2}^2 - r_{1}^2 + x_{1}^2 - x_{2}^2 + y_{1}^2 - y_{2}^2 \\
      r_{3}^2 - r_{1}^2 + x_{1}^2 - x_{3}^2 + y_{1}^2 - y_{3}^2
    \end{pmatrix}
    $
  \end{center}
\end{frame}

\begin{frame}
  \frametitle{Algorithmen}

  \begin{itemize}
  \item Maximum likelihood
    \begin{itemize}
    \item gut auf µC zu berechnen
    \item kaum Möglichkeiten die \textit{a priori}
      Wahrscheinlichkeiten für alle Messungen zu erhalten
    \end{itemize}
  \item Erweiterter Kalman-Filter
    \begin{itemize}
    \item durch Prädiktion und Korrektur gute Ergebnisse
    \item findet Einsatz in der Praxis (nanoLOC)
    \end{itemize}
  \end{itemize}
\end{frame}

\begin{frame}
  \frametitle{Kalman-Filter Test}

  \begin{center}
    \includegraphics[scale=0.5]{img/kalman}
  \end{center}
\end{frame}

\begin{frame}
  \frametitle{Semidefinite Optimierung}

  \begin{itemize}
  \item geometrische Beziehungen $k$ der Knoten werden in linearen Matrizen
    dargestellt
  \item diese Form der Darstellung ist aber nicht in allen Fällen
    möglich
  \item der Algorithmus zur Lösung des Systems hat eine hohe Laufzeit
    \begin{itemize}
    \item $O(k^2)$ für AoA
    \item $O(k^3)$ wenn Hop Counts verwendet werden
    \end{itemize}
  \item kaum relevant in echten Anwendungen, da die Laufzeit zu
    schlecht
  \end{itemize}
\end{frame}

\begin{frame}
  \frametitle{Weitere Algorithmen}

  \begin{itemize}
  \item Nonlinear least squares
  \item Linear least squares
  \item Second-order cone programming
  \item Projections onto convex sets
  \item Bounding the feasible set
  \end{itemize}
\end{frame}