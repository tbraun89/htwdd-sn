\section{Messmethoden}
\subsection{Anchor\/Beacon Nodes}

Als \textit{Anchor- oder Beacon Nodes} werden die Sensorknoten bezeichnet, deren Position in einem globalen Koordinatensystem bekannt sind.
Diese Information kann auf zwei Weisen gewonnen werden. Zum ersten ist es möglich, die Sensorknoten die als Ankerknoten fungieren sollen, 
mit einem GPS-Chip auszustatten und auf diese Weise die Position zu jeder beliebigen Zeit zu ermitteln. Ein solcher Ansatz ist besonders
für den Fall sinnvoll, wenn der Ankerknoten oder sogar das gesammte Sensornetz die Möglichkeit haben soll mobil zu sein. 
Die andere Möglichkeit besteht darin, die exakte Position des Ankerknotens fest einzuprogrammieren. Solch ein Ansatz ist empfehlenswert,
sollte das Sensornetz oder zumindest die Ankerknoten statisch sein. Der große Vorteil dieser Methode der vorprogrammierten Koordinaten ist,
das keinerlei zusätzlich Hardware am Sensorknoten angebracht werden muss und somit keine weiteren finanziellen Kosten anfallen und ebenso der
Energiebedarf der Knoten niedrig gehalten werden kann. \\~\\
Um nun die Ankerknoten effektiv nutzen zu können benötigt man zur Erstellung eines globalen zweidimensionalen Koordinatensystems mindestens
drei Ankerknoten, welche nicht linear angeordnet sein dürfen. Soll sogar ein dreidimensonales globales Koordinatensystem erzeugt werden, wird
ein weitere Ankerknoten benötigt. Hier gilt die Einschränkung des zweidimensionalen Falles und zusätzlich muss der vierte Ankerknoten auf einer
anderen Ebene sein, als die anderen drei. Als sehr effektiv gilt hier die Anordnung als Tetraeder. \\~\\
Um nun unter Zuhilfenahme der Ankerknoten ein Koordinatensystem aufbauen zu können, gibt es zwei Möglichkeiten. Zum einen ist es realisierbar das 
ein mit Hilfe aller Sensorknoten erzeugtes relatives Koordinatensystem, nachträglich auf Ankerknoten gelegt werden und somit die relativen Positionen, 
in globale umrechenbar sind. Andererseits ist es möglich die bekannten Positionen direkt bei den Messungen zu nutzen und so, jede Position eines
Sensorknotens direkt als globale zu errechnen. 

\subsection{Signalstärkemessung}
Die Signalstärkemessung oder \ac{RSSI} basiert auf der Idee, die bei einem kabellosen Sensornetz vorhandenen Sender und Empfänger zu nutzen um die 
Entfernung von Sensorknoten untereinander zu messen. Es wird davon ausgegangen das alle Sensorknoten eine identische Sendeleistung haben und es nun
möglich sein sollte über die ankommende Signalstärke exakt zu berechnen, wie weit der emittierende Sensorknoten vom empfangenden entfernt ist. Als 
physikalische Grundlage dient an dieser Stelle, das die Signalstärke im Vakuum, quadratisch zur zurückgelegten Entfernung abnimmt. In der Praxis 
hat sich allerdings herausgestellt das die Signalstärke durch aller Hand Störfaktoren, wie Luftfeuchtigkeit und -temperatur aber auch Störstrahlung 
und Hindernisse welche das Signal reflektiven oder absorbieren, beeinflusst wird. Die folgende Abbildung zeigt das Ergebnis einer Messreihe und es 
wird hier deutlich, dass das reale Messergebnis in keiner Weise mit dem idealen kreisförmigen Ergebnis der Theorie übereinstimmt. 

\begin{figure}
  \caption{RSSI Meassurement}
  \includegraphics[scale=0.75]{img/RSSI1}\\
  \cite{whitehouse}
\end{figure}

\subsection{Hop Count}
blabla

